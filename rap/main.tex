\documentclass[a4paper,12pt]{article}
\usepackage[utf8]{inputenc}
\usepackage[polish]{babel}
\usepackage{amsmath}
\usepackage{amssymb}
\usepackage{listings}
\usepackage{graphicx}
\usepackage{geometry}
\usepackage{adjustbox}
\usepackage[T1]{fontenc}
\geometry{margin=1in}
\title{Układy cyfrowe i systemy wbudowane \newline Projekt czujnika temperatury}
\author{Kornel Uriasz 272967 \newline Jakub Maciocha <mehiko>}
\date{18.01.25r}

\begin{document}

\maketitle

\section{Cel projektu}
Celem projektu jest zaprojektowanie systemu pomiaru temperatury, który wykorzystuje czujnik MAX31856 i termoparę typu K. System sterowany jest przez układ FPGA Arty Z7-20 i prezentuje zmierzoną temperaturę na wyświetlaczu 7-segmentowym.

\section{Komponenty projektu}
\begin{itemize}
    \item \textbf{FPGA Arty Z7-20}: Układ FPGA realizujący komunikację SPI z czujnikiem i sterujący wyświetlaczem 7-segmentowym.
    \item \textbf{Adafruit MAX31856}: Precyzyjny czujnik temperatury obsługujący różne typy termopar, komunikujący się przez interfejs SPI.
    \item \textbf{Termopara typu K}: Czujnik temperatury generujący sygnał napięciowy proporcjonalny do temperatury.
    \item \textbf{Wyświetlacz 7-segmentowy}: Urządzenie wyświetlające zmierzoną temperaturę w formacie liczbowym.
    \item \textbf{Zasilanie i okablowanie}: Dodatkowe elementy niezbędne do funkcjonowania systemu.
\end{itemize}

\section{Architektura systemu}
System składa się z następujących modułów:
\begin{enumerate}
    \item \textbf{Moduł SPI}: Realizuje komunikację między FPGA a czujnikiem ADAFRUIT MAX31856, obsługując wysyłanie i odbieranie danych.
    \item \textbf{Moduł przetwarzania danych}: Konwertuje 2-bajtowe dane odczytane z czujnika na wartość temperatury w BCD , uwzględniając kompensację zimnego końca.
    \item \textbf{Sterownik wyświetlacza 7-segmentowego}: Zarządza wyświetlaniem wartości temperatury, konwertując dane na format dziesiętny.
    \item \textbf{Główna maszyna stanów}: Realizuje logikę sterującą, cykliczny odczyt danych i ich wyświetlanie.
\end{enumerate}

\section{Schemat działania}
\begin{enumerate}
    \item \textbf{Pomiar temperatury}: Czujnik MAX31856 odczytuje sygnał z termopary typu K i konwertuje go na wartość cyfrową za pomocą wbudowanego przetwornika ADC.
    \item \textbf{Odczyt danych przez FPGA}: FPGA inicjuje komunikację przez SPI i odczytuje rejestry z wartością temperatury.
    \item \textbf{Przetwarzanie danych}: Dane z czujnika są konwertowane na postać dziesiętną (np. 25\(^\circ C\)).
    \item \textbf{Wyświetlanie temperatury}: FPGA steruje wyświetlaczem 7-segmentowym, prezentując zmierzoną wartość w czasie rzeczywistym.
\end{enumerate}

\section{Opis kodu transmisji SPI}

\subsection*{Wejścia Wyjścia}
Moduł \texttt{spi.sv} odpowiada za implementację interfejsu SPI w trybie master, umożliwiającego komunikację z czujnikiem temperatury. Główne funkcje modułu obejmują:

\begin{itemize}
    \item Generowanie sygnałów zegarowych SPI (SCK).
    \item Obsługę linii danych MOSI (Master Out Slave In) i MISO (Master In Slave Out).
    \item Zarządzanie sygnałem wyboru układu (CS\_N).
    \item Obsługę rejestrów przesuwających do wysyłania i odbierania danych.
\end{itemize}

Moduł przyjmuje następujące wejścia i wyjścia:
\begin{itemize}
    \item \textbf{Wejścia}:
    \begin{itemize}
        \item \texttt{clk}: Główny zegar systemowy FPGA.
        \item \texttt{reset}: Sygnał resetujący moduł.
        \item \texttt{in\_bytes}: Dane do wysłania przez SPI.
        \item \texttt{in\_bytes\_count}: Dane do wysłania przez SPI.
        \item \texttt{trans\_beg}: Sygnał rozpoczynający transmisję.
        \item \texttt{miso}: Dane odbierane z urządzenia slave.
    \end{itemize}
    \item \textbf{Wyjścia}:
    \begin{itemize}
        \item \texttt{out\_bytes\_count}: Otrzymane dane z urządzenia slave.
        \item \texttt{trans\_done}: Flaga sygnalizująca finalizacje transakcji.
        \item \texttt{cs\_n}: Sygnał wyboru układu.
        \item \texttt{sck}: Sygnał zegarowy SPI.
        \item \texttt{mosi}: Dane wysyłane do urządzenia slave.
    \end{itemize}
\end{itemize}

\subsection*{Inicljalizacja stanów wewnętrznych}
\begin{lstlisting}[language=verilog]
initial begin
    cs            <= 1'b1;
    sck_out       <= 1'b1;
    mosi          <= 1'b0;

    out_bytes     <= 32'h00000000;

    old_start     <= 1'b0;

    bit_counter   <= 3'b111;

    cur_state     <= idle;
    next_state    <= idle;

    in_bytes_counter  <= 0;
    out_bytes_counter <= 0;

    trans_done    <= 1'b1;

    en_out        <= 1'b1;
end
\end{lstlisting}

W tej sekcji zwyczajnie inicjalizujemy wszystkie sygnaly we/wy oraz wewnętrzne na wartości domyślne.

\subsection*{Maszyna stanów transakcji SPI}

\begin{lstlisting}[language=verilog]
  localparam idle         = 3'b000;
  localparam trans_write  = 3'b001;
  localparam trans_read   = 3'b010;
  localparam trans_beg    = 3'b011;
  localparam trans_end    = 3'b100;

  reg [2:0] cur_state;
\end{lstlisting}

Maszyna stanów transakcji SPI ma 5 dozwolonych stanów. Jest to 3-bitowa maszyna stanów, ze stanami takimi jak:

\begin{itemize}
  \item \textbf{idle} : Stan czekania na rozpoczęcie transmisji SPI. Podczas zmiany stanu logicznego sygnału trans\_beg zmieniamy stan wewnętrzny na trans\_beg, oraz zaczytujemy do rejestrów wewnętrznych długość transmisji na MISO oraz MOSI.

\begin{lstlisting}[language=verilog]
 idle :
    begin
      if (old_start != start_trans) begin
        old_start     <= start_trans;
        next_state    <= trans_beg;

        in_bytes_counter    <= in_bytes_count;
        out_bytes_counter   <= out_bytes_count;

        trans_done          <= 1'b0;
      end

      cs <= 1'b1;
    end
  trans_beg :
\end{lstlisting}


  \item \textbf{trans\_beg} : Stan rozpoczęcia transakcji. W tym stanie sterujemy sygnał CS na stan logiczny 0, oraz sygnał wewnętrzny en\_out również jest ustawiany na 0. Sygnał ten jest używany do włączenia zegara na pinie sck\_out. Stan wewnętrzny programu jest tez zmieniany na stan trans\_write.

\begin{lstlisting}[language=verilog]
 assign sck_out = ((en_out == 1'b0) ? sck_in : 1'b1);

 ...

 trans_beg :
      begin
        cs          <= 1'b0;
        en_out      <= 1'b0;

        next_state  = trans_write;
      end
\end{lstlisting}

  \item \textbf{trans\_write} : Stan wypisywania danych na szyne MOSI. W tym stanie po kolei wypisujemy dane zaczynając od MSB najbardziej znaczącego bajtu. Używamy do tego wewnętrznych stanów in\_bytes\_counter oraz bit\_count. Po zakończeniu transmisji zmieniamy stan na trans\_read.

\begin{lstlisting}[language=verilog]
trans_write :
    begin
      if (in_bytes_counter == 0) begin
        if (out_bytes_counter == 0) begin
          bit_counter = 3'b111;

          next_state = trans_end;
        end else begin
          bit_counter = 3'b111;

          next_state = trans_read;
        end
      end else begin
        if (bit_counter == 3'b000) begin
          bit_counter = 3'b111;

          in_bytes_counter = in_bytes_counter - 1;

          if (in_bytes_counter == 0) begin
            bit_counter = 3'b111;

            next_state = trans_read;
          end
        end else begin
          bit_counter = bit_counter - 1;
        end
      end
    end
\end{lstlisting}

Warto zauważyć, że wszystkie powyższe operacje odbywają się w bloku always, który reaguje na zbocze narastające sygnału sck\_in. Pod koniec tego bloku aktualizujemy stan wewnętrznej maszyny stanów.

\begin{lstlisting}[language=verilog]

always @(posedge sck_in) begin
  begin
    case (cur_state):

    ...

    endcase

    cur_state <= next_state.
  end
end

\end{lstlisting}

Jedyne operacje reagujące na zbocze opadające to samo wypisywanie danych na szyne MOSI. Musi tak być, aby stan był stabilny podczas zbocza narastającego sygnału sck\_in, w którym to są zczytywane dane przez obie strony transakcji.

\begin{lstlisting}[language=verilog]
always @(negedge sck_in) begin
  if (cur_state == trans_write) begin
    mosi <= in_bytes[(in_bytes_counter - 1)*8 + bit_counter];
  end
end
\end{lstlisting}

  \item \textbf{trans\_read} : Stan zczytywania danych z szyny MISO. W tym stanie po kolei zczytujemy dane zaczynając od MSB najbardziej znaczącego bajtu. Używamy do tego wewnętrznych stanów out\_bytes\_counter oraz bit\_count. Po zakończeniu transmisji zmieniamy stan na trans\_end.

\begin{lstlisting}[language=verilog]
trans_read :
    begin
      out_bytes[(out_bytes_counter - 1)*8 + bit_counter] <= miso;
      if (out_bytes_counter == 2'b00) begin
        next_state  = trans_end;
      end else begin
        if (bit_counter == 3'b000) begin
          bit_counter = 3'b111;
          out_bytes_counter = out_bytes_counter - 1;
        end else begin
          bit_counter = bit_counter - 1;
        end
      end
    end
\end{lstlisting}

  \item \textbf{trans\_end} : Stan kończenia transmisji, w którym ustawiamy następny stan na idle oczekując na rozpoczęcie kolejnej transmisji. Ustawiamy także sygnał en\_out na 1, aby wyłączyć sygnął zegarowy, oraz sygnał trans\_done aby zasygnalizować zakończenie transmisji. Wysterowanie sygnału cs na stan 1 następuje już w stanie idle, ponieważ specyfikacja hardware-owa sterownika termopary oczekuje offsetu czasowego między wysterowaniem sygnału sck\_out na 1 oraz cs na 1.

\begin{lstlisting}[language=verilog]
trans_end :
     begin
       if (sck_in == 1'b1) begin
         next_state = idle;

         en_out     <= 1'b1;
         trans_done <= 1'b1;
       end
     end
  endcase
\end{lstlisting}
\end{itemize}

\section{Opis kodu oraz stanów Głównej maszyny stanów}

\subsection*{Opis wejść i wyjść}

\begin{lstlisting}[language=verilog]
    module top
  (
    input btn_0,
    input sys_clk_pin,
    input phy_miso,

    output led_0,
    output phy_sck,
    output phy_mosi,
    output phy_cs,
    output [2:0]led,

    output [3:0] control_out_pins,
    output [6:0] display_out_pins
  );
\end{lstlisting}

Moduł przyjmuje następujące wejścia i wyjścia:
\begin{itemize}
    \item \textbf{Wejścia}:
    \begin{itemize}
        \item \texttt{sys\_clk\_pin}: Wygenerowany sygnał zegarowy.
        \item \texttt{phy\_miso}: Pin MISO.
    \end{itemize}
    \item \textbf{Wyjścia}:
    \begin{itemize}
        \item \texttt{led\_0}: Sygnalizuje przez miganie leda rozpoczęcie transmisji.
        \item \texttt{phy\_sck}: Pin SCK.
        \item \texttt{phy\_mosi}: Pin MOSI.
        \item \texttt{phy\_cs}: Pin CS.
        \item \texttt{led[2:0]}: Ledy na płytce sygnalizujące aktualny stan głównej maszyny stanów.

        \item \texttt{[3:0] control\_out\_pins}: display <mehiko>
        \item \texttt{[6:0] display\_out\_pins}: display <mehiko>
    \end{itemize}
\end{itemize}

\subsection*{Inicjalizacja sygnałów wewnętrznych i zewnętrznych}

\begin{lstlisting}[language=verilog]
initial begin
  in_bytes_count    =     2'b01;
  out_bytes_count   =     2'b01;
  in_bytes          =     8'h01;
  led_reg           =     1'b0;
  cur_state         = check_con;
  en_dis            = 1'b0;
end
\end{lstlisting}

\subsection*{Stany głównej maszyny stanów}

\begin{lstlisting}[language=verilog]
localparam check_con     = 3'b000;
localparam set_options   = 3'b001;
localparam chk_options   = 3'b010;
localparam read_temp     = 3'b011;
\end{lstlisting}

Maszyna stanów transakcji SPI ma 4 dozwolonych stanów. Jest to 2-bitowa maszyna stanów, ze stanami takimi jak:

\begin{itemize}
  \item \textbf{check\_con} : W tym stanie wysyłamy poprzez interfejs SPI do czujnika termopary prośbe o sczytanie rejestru o znanej zawartości, w celu sprawdzenia poprawności transmisji. Kiedy Wartość oczekiwana jest otrzymana  przechodzimy do stanu set\_options.

\begin{lstlisting}[language=verilog]
check_con:
      begin
        if (out_bytes[7:0] == 8'h03) begin
          cur_state   <= set_options;
          in_bytes_count        <= 3;
          out_bytes_count       <= 0;
          in_bytes[7:0]         <= 8'h03;
          in_bytes[15:8]        <= 8'h81;
          in_bytes[23:16]       <= 8'h80;
        end
        led_reg     <= ~led_reg;
      end
\end{lstlisting}

  \item \textbf{set\_options oraz chk\_options} : W tych stanach wysyłamy poprzez interfejs SPI do czujnika termopary konfiguracje, w celu ustawienia typu termopary na typ K. Następnie sprawdzamy czy dane zostały zapisane w rejestrach i przechodzimy do stanu read\_temp.

\begin{lstlisting}[language=verilog]
set_options:
      begin
          in_bytes_count        <= 3;
          out_bytes_count       <= 0;
          in_bytes[7:0]         <= 8'h03;
          in_bytes[15:8]        <= 8'h81;
          in_bytes[23:16]       <= 8'h80;
          led_reg               <= 1'b1;
          cur_state             <= chk_options;
      end
    chk_options:
      begin
          in_bytes_count        <= 2'b01;
          out_bytes_count       <= 2'b10;
          in_bytes[7:0]         <= 8'h00;
          if (out_bytes[15:8] == 8'h81 && out_bytes[7:0] == 8'h03) begin
            cur_state <= read_temp;
            en_dis    <= 1'b1;
          end
      end
\end{lstlisting}

  \item \textbf{read\_temp} : W tym stanie w kółko sczytujemy temperature z rejestru temperatury termopary.

\begin{lstlisting}[language=verilog]
 read_temp:
      begin
          in_bytes_count        <= 2'b01;
          out_bytes_count       <= 2'b10;
          in_bytes[7:0]         <= 8'h0C;
          if (out_bytes[15:8] == 8'h81 && out_bytes[7:0] == 8'h03) begin
            cur_state <= read_temp;
          end
      end
\end{lstlisting}

<mehiko> dokoncz tu prosze koncowke opisu tego, czyli sygnaly w inicjalizacji roznych modulow.

\end{itemize}

\section{Kod transofmatora na BCD <mehiko>}

\section{Kod wyświetlacza 7-seg <mehiko>}

\section{Testowanie i debugowanie}
\begin{itemize}
    \item Przeprowadzenie symulacji funkcjonalnych modułów (np. SPI, wyświetlacza). \newline
    Wartość takich testów niestety jest niska, ponieważ testy przechodzące w symulacji nie mają żadnego pokrycia w rzeczywistym sprzęcie, często nie potrafiąc nawet dokonać syntezy albo implementacji. Przykładowy test:

\begin{lstlisting}[language=verilog]
 module presc_tb();

reg clk_in;
wire clk_out;

presc presc_0(
               .clk_in_p(clk_in),
               .clk_out_p(clk_out)
             );

integer i;

initial begin
  clk_in = 1'b0;
  $monitor("time:%d clk_out:%b", $time, clk_out);
  test_presc();
  $finish;
end

task test_presc();
  begin
    for(i=0; i<256; i=i+1) begin
      clk_in = ~clk_in;
      #1;
      clk_in = ~clk_in;
      #1;
    end
  end
endtask

endmodule

\end{lstlisting}

    \item Testowanie systemu na fizycznej płytce FPGA z podłączonym analizatorem stanów logicznych.
    Wartość takich testów jest już jak najbardziej odczuwalna. Po kolei łącząc kolejne małe części programu jesteśmy w stanie odczytywać fizyczne wartości pinów : <mehiko> wklej tu prosze fote z analizatora.
    \item Testowanie systemu na fizycznej płytce FPGA z podłączonym czujnikiem i wyświetlaczem.
    <mehiko> tu bedzie trzeba wrzucic zdjecie, pingnij mnie
\end{itemize}

\section{Moduły poboczne}

Do realizacji powyższego zadania potrzebne były pomniejsze moduły, przez specyfikacje hardware-ową czujnika termopary. Są to :

\begin{itemize}
  \item \textbf{prescaler} : Służy do zmniejszania częstotliwośći sygnału zegarowego:
  \begin{lstlisting}[language=verilog]
    module presc(
              input wire clk_in_p,
              output reg clk_out_p
            );

  parameter value = 65535;
  
  localparam bit_count = $clog2(value);
  
  reg [bit_count:0]count;
  
  initial begin
    count       <= 0;
    clk_out_p   <= 1'b0;
  end
  
  always @(posedge clk_in_p) begin
    if(count < (value/2)) begin
      count = count + 1;
    end else begin
      count = 0;
      clk_out_p = ~clk_out_p;
    end
  end
  
  endmodule
  \end{lstlisting}

  \item \textbf{timer} : Służy do zmiany stanu na wyjściu po określonej ilości czasu.
  \begin{lstlisting}[language=verilog]
  module timer (
              input wire clk_in,
              output reg sig_out
             );

parameter time_ms = 1000;

localparam clk_per_ms = 180000;
localparam reset_count = (clk_per_ms*time_ms);
localparam bit_count = $clog2(reset_count);

reg [bit_count:0]counter;

initial begin
  counter <= 0;
  sig_out <= 1'b0;
end

always @(posedge clk_in) begin
  if (counter < reset_count) begin
    counter = counter + 1;
  end else begin
    counter = 0;
    sig_out = ~sig_out;
  end
end

endmodule
  \end{lstlisting}
  \item \textbf{counter} : Służy do liczenia w góre, zwiększając licznik o jeden co zadany kwant czasu.
  \begin{lstlisting}[language=verilog]
  module counter(
    input wire clk,
    input wire rst,
    output reg [23:0] count
    );

    initial begin
        count[23:0] <= 24'h000;
    end

    always @(posedge clk or posedge rst) begin
        if (rst) begin
            count <= 24'h000; // Reset count to 0
        end else if (count == 24'd999900) begin
            count <= 24'h000;
        end else begin
            count <= count + 1; // Increment count
        end
    end

endmodule
  \end{lstlisting}
\end{itemize}

\section{Wynik końcowy}
Zaprojektowany system wyświetla zmierzoną temperaturę w czasie rzeczywistym na wyświetlaczu 7-segmentowym, zapewniając precyzyjne i stabilne pomiary dzięki wbudowanej kompensacji i filtracji danych.

\end{document}
